\documentclass{ees}

\begin{document}

\eesTitlePage

\eesCriticalReport{
  – & – & 2 trb and 2 vla should play unison with A and T, respectively, in tutti sections, as explicitly stated by notes in the \textit{Kyrie} (“Viole e Tromboni si caneranno dalle parti”), \textit{Kyrie II} (“i V.V. Viole e Tromboni con le parti”), \textit{Cum Sancto} (“li strumenti dalle parti”). \\
  – & – & In several movements, the da capo is indicated by directives such as “Da Capo sine al segno” in \A1, but written out here: \textit{Kyrie II}, second half of bar 76; \textit{Gratias}, 2nd \quarterNote\ of bar 96; \textit{Domine Deus}, bar 194; \textit{Qui tollis}, bar 281 to 288; \textit{Quoniam}, bar 347 to 355;
  1 & 71 & S & 2nd/3rd \eighthNote\ in \A1: d″16–e″16–f″8 \\
  2 & 62 & vl & 3rd \eighthNote\ in \A1: e″8 \\
    & 83 & vl 2 & 1st \halfNote\ in \A1: a′4–a′8–b′8 \\
    & 219–235 & – & \A1 only contains staves for four voices and org, as well as the directive “li strumenti dalle parti”. \\
  3 & 157 & tr 2 & 1st \quarterNote\ in \A1: c′4 \\
  4 & 31 & tr 2 & 1st \halfNote\ in \A1: g2 \\
  5 & 16–50 & In \A1, the \textit{Dona nobis} is indicated by the directive “Il finale del Chirie servizi per il Dona nobis Pacem”. Hence, all lyrics in this section have been added by the editor. Moreover, bars 34 (trb 1, vl 2, A), 46 (trb 2, T, B), and 49 (trb, vl, all voices) were slightly adopted to accomodate the new lyrics. \\
}

\eesToc{}

\eesScore

\end{document}
